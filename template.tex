\section{Mein erster Abschnitt}

\subsection{Mein erster Unterabschnitt}

\subsubsection{Mein erster Unter-Unterabschnitt}

\section{Ein paar wichtige Befehle um Mathematik zu schreiben}

Mathematische Notation wird z.B. durch das \$ (Dollarzeichen) eingeschlossen: 

Dies ist eine Formel $x^2$ inline, d.h., innerhalb des Textflusses welche durch \$ ... \$ nur je durch ein \$-Zeichen eingeschlossen wird (siehe Quellcode).

Und dies ist eine $$f_1 = g_2$$ vom Textfluss abgesetzte Formel, welche in zwei \$\$ ... \$\$ eingeschlossen wird. Will man eine Formel automatisch nummerieren lassen, dann kann man das so tun:

\begin{align}
	f(x) &:= 2 \cos(x) 
\end{align}

Das \&-Zeichen im LaTeX-Quellcode regelt hier die Ausrichtung bei mehreren Zeilen, zwei Backslashes markieren den Zeilenumbruch in der ''align''-Umgebung:

\begin{align}
	f(x) &= 2 \cos(x) \\
	g(x) &= 3 \sin(x)
\end{align}		

Formeln ausrichten, aber ohne Nummerierung geht mit dem align-Stern-Befehl im LaTeX-Quellcode:

\begin{align*}
	f(x) &= \alpha \cos(x) \\
	g(x) &= \beta \sin(x)
\end{align*}		

\section{Ein paar wichtige schnelle Textformatierungen}

Das ist ein \textcolor{red}{roter Text}. Dies ist ein {\itshape kursiver Textteil} und dies ist ein {\bfseries fetter Textteil}. 

\subsection{Unnumerierte Liste}

\begin{itemize}
	\item Dies ist der erste Punkt einer Liste.
  	\item Dies der zweite Punkt.
  	\item Solche Listen können auch geschachtelt werden.
\end{itemize} 

\subsection{Nummerierte Liste}

\begin{enumerate}
	\item Dies ist der erste Punkt einer nummerierten Liste.
	\item Dies der zweite Punkt.
	\item Auch solche Listen können geschachtelt werden.
\end{enumerate}  		


\section{Eine einfache Tabelle}

\begin{center}
	\begin{tabular}{ccc}
		a & b & d	\\ \hline
		d & f & e
	\end{tabular}
\end{center}

Die \&-Zeichen markieren die Spalten, zwei Backslashes markieren Zeilenumbrüche. Das Option \texttt{ccc} im Quellcode besagt ''drei Spalten, jede ''centered ausgerichtet''. 

\section{Ausblick}

\begin{itemize}
	\item Eine sehr umfassende Liste von LaTeX-Symbolen: \\
		\href{http://tug.ctan.org/info/symbols/comprehensive/symbols-a4.pdf}{http://tug.ctan.org/info/symbols/comprehensive/symbols-a4.pdf}
	\item LaTeX-Project: \\
		\href{https://www.latex-project.org}{https://www.latex-project.org}
\end{itemize}	
		
Es gibt noch sehr, sehr viel mehr und es lassen sich viele Dinge auch sehr praktisch automatisieren. Siehe LaTeX-Manuals, Google-Suche oder Frage an mich :-)
