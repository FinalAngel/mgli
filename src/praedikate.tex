\documentclass[../gruppenarbeit_1.tex]{subfiles}
\begin{document}
    \section{Prädikatenlogik}
    \textbf{Was sind Prädikate?}
    \\
    \begin{align}
        \fbox{\begin{minipage}{11.8cm}
                  \textbf{Beispiel 1}: Bsp. Aussage über ein mathematisches Objekt: $A≔5$ ist eine ungerade Zahl.
        \end{minipage}}
    \end{align}
    \\Eine Aussage bestand bisher immer daraus, etwas über die Eigenschaft eines Objekts zu sagen. Solche Aussagen werden in der Mathematik «Prädikate» genannt.
    \\
    \\Die Zuordnung eines Prädikats wird wie folgt dargestellt: $P(x)$. $P$ ist das Prädikat, wobei $x$ das Objekt ist.
    \\
    \\\textbf{Definintion von verschiedenen Prädikaten}
    \\
    \\Enthält ein Prädikat mehrere Objektvariablen, so nennen wir unser Prädikat «n-stelliges Prädikat» oder eine «n-stellige Aussageform». Wenn jedoch eine Aussage keine Variable enthält, reden wir von einem «0-stelligen Prädikat».
    \begin{align}
        \fbox{\begin{minipage}{11.8cm}
                  \textbf{Beispiel 1}: n-stelliges Prädikat: $Q(x,\ y) :=$ «x ist Automarke und y ist das Modell». Das Prädikat $Q(VW,\ Golf)$ ist wahr, wobei $Q(Toyota,\ Polo)$ falsch ist.
        \end{minipage}}
    \end{align}
    \\
    \begin{align}
        \fbox{\begin{minipage}{11.8cm}
                  \textbf{Beispiel 2}: $V(3)∶=$ «3 ist eine ungerade Zahl» ist eine Aussage und somit ein 0-stelliges Prädikat.
        \end{minipage}}
    \end{align}
    \\\textbf{Äquivalenz von zwei Prädikaten}
    \\
    \\Wenn alle Objektvariablenbelegungen dieselben Wahrheitswerte haben, dann wird das folgendermassen dargestellt: $P(x) \iff Q(x) oder P(x) \Rightarrow Q(x) und Q(x) \Rightarrow P(x)$
    \begin{align}
        \fbox{\begin{minipage}{11.8cm}
                  \textbf{Beispiel}: $P(x)∶=$ «$x$ ist durch 20 teilbar» und $Q(x)∶=$ «$x$ ist durch 4 und 5 teilbar». Dann ist $P(x) <==> Q(x)$.
        \end{minipage}}
    \end{align}
    \\Jedoch nicht, wenn: $P(x)$ eine Primzahl ist und $Q(x)$ eine ungerade natürliche Zahl ist.
    \\
    \\\textbf{Quantoren}
    \\
    \def\arraystretch{1.5}
    \begin{table}[ht]
        \begin{tabular}[t]{ll}
            \hline
            Kürzel & Definition\\
            \hline
            $\forall x$ & Für alle x\\
            $\exists x$ & Es gibt mindestens ein x\\
            $\exists!$ & Es gibt nur ein x\\
            \hline
        \end{tabular}
    \end{table}
    \\
    \\Wenn $P$ irgendein Prädikat ist und $x$ eine Objektvariable einer bestimmten Objektklasse, dann gilt: «für alle $x$ gilt $P(x)$» die wird folgendermassen geschrieben: $\forall x:P(x)$
    \\
    \\Das Symbol $\forall$ wird auch Allquantor oder Generalisator oder Generalisator-Operator genannt.
    \\
    \\Wenn es mindestens ein $x$ gibt, also: «es gibt (mind.) ein $x$, so dass $P(x)$ gilt», dann schreibt man: $\exists x:P(x)$
    \\
    \\Das Symbol $\exists$ wird Existenzquantor oder Partikularisator genannt.
    \\
    \\Wenn es jedoch genau nur ein Objekt gibt, dann: «es gibt genau ein $x$, so dass $P(x)$ gilt». Dies wird wie folgt dargestellt: $\exists! x:P(x)$.
    \\
    \\Das Symbol $\exists!$ wird Eindeutigkeitsquantor oder Einzigkeitsquantor genannt.
    \\
    \\\textbf{Ausserdem: Ein Quantor bindet stärker als alle anderen logischen Operatoren!!!}
    \\
    \begin{align}
        \fbox{\begin{minipage}{11.8cm}
                  \textbf{Beispiel für die Verneinung}: $\neg \forall xPx\leftrightarrow \exists x\neg Px$: Die Aussage: Alle Wände sind weiss läst sich auch als "Nicht alle Wände sind weiss", und als "Es gibt mindestens eine Wand, die nicht weiss ist".
        \end{minipage}}
    \end{align}
    \def\arraystretch{1.5}
    \begin{table}[ht]
        \begin{tabular}[t]{ll}
            \hline
            Distributivgestze der Prädikate\\
            \hline
            $\forall x(Px\wedge Qx)\leftrightarrow (\forall xPx\wedge \forall xQx)$\\
            ${\displaystyle \exists x(Px\lor Qx)\leftrightarrow (\exists xPx\lor \exists xQx)}$\\
            \hline
        \end{tabular}
    \end{table}
\end{document}
