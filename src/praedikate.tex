\documentclass[../gruppenarbeit_1.tex]{subfiles}

\subsection{Was sind Prädikate?}

Eine Aussage bestand bisher immer daraus, etwas über die Eigenschaft 
eines Objekts zu sagen. Solche Aussagen werden in der Mathematik «Prädikate» 
genannt.
\vspace{1mm}

Die Zuordnung eines Prädikats wird wie folgt dargestellt: $P(x)$. 
$P$ ist das Prädikat, wobei $x$ das Objekt ist.

\begin{align}
\fbox{\begin{minipage}{11.8cm}
    \textbf{Beispiel}: Aussage über ein mathematisches Objekt: $A:=5$ ist 
    eine ungerade Zahl.
\end{minipage}}
\end{align}
    
\subsection{Definintion von verschiedenen Prädikaten}

Enthält ein Prädikat mehrere Objektvariablen, so nennen wir unser Prädikat 
«n-stelliges Prädikat» oder eine «n-stellige Aussageform». Wenn jedoch eine 
Aussage keine Variable enthält, reden wir von einem «0-stelligen Prädikat».

\begin{align}
\fbox{\begin{minipage}{11.8cm}
    \textbf{Beispiel}: n-stelliges Prädikat:\\
    $Q(x,\ y) :=$ «x ist Automarke und y ist das Modell».\\
    Das Prädikat $Q(VW,\ Golf)$ ist wahr, wobei $Q(Toyota,\ Polo)$ falsch ist.
\end{minipage}}
\end{align}

\begin{align}
\fbox{\begin{minipage}{11.8cm}
    \textbf{Beispiel}: $V(3):=$ «3 ist eine ungerade Zahl» ist eine Aussage 
    und somit ein 0-stelliges Prädikat.
\end{minipage}}
\end{align}

\subsection{Äquivalenz von zwei Prädikaten}

Wenn alle Objektvariablenbelegungen dieselben Wahrheitswerte haben,\\
dann wird das folgendermassen dargestellt: $P(x) \iff Q(x)$ oder 
$P(x) \Rightarrow Q(x) und Q(x) \Rightarrow P(x)$

\begin{align}
\fbox{\begin{minipage}{11.8cm}
    \textbf{Beispiel}: $P(x):=$ «$x$ ist durch 20 teilbar» und\\
    $Q(x):=$ «$x$ ist durch 4 und 5 teilbar». Dann ist $P(x) \iff Q(x)$.
\end{minipage}}
\end{align}

Jedoch nicht, wenn: $P(x)$ eine Primzahl ist und $Q(x)$ eine ungerade 
natürliche Zahl ist.

\subsection{Quantoren}

\def\arraystretch{1.5}
\begin{table}[ht]
\begin{tabular}[t]{lll}
    \hline
    Kürzel & Definition\\
    \hline
    $\forall x$ & Für alle $x$\\
    $\exists x$ & Es gibt mindestens ein $x$\\
    $\exists!$ & Es gibt nur ein $x$\\
    \hline
\end{tabular}
\end{table}

\begin{align}
\fbox{\begin{minipage}{11.8cm}
    \textbf{Beispiele}:\\
    $\forall x$\\
    Bezeichnet x ein Objekt der Gesamheit aller Katzen. Dann ist 
    $\forall x$ : "$x$ hat Schnurrhaare.".\\
    \\
    $\exists x$\\
    Bezeichnet $a$ die Personen in unserem Unterrichtsraum. 
    Dann ist $\exists! a$ : "$a$ ist Dozent." eine prädikatenlogische 
    Aussage.)\\
    \\
    $\exists!$\\
    Bezeichnet $m$ alle Menschen auf unserem Planeten. 
    $\exists m : m ist Kannibale"$ heisst, es gibt unter allen Menschen auf 
    diesem Planeten sicher einen Kannibalen.\\
\end{minipage}}
\end{align}

\subsubsection{Allquantor}

Wenn $P$ irgendein Prädikat ist und $x$ eine Objektvariable einer bestimmten 
Objektklasse, dann gilt: «für alle $x$ gilt $P(x)$» die wird folgendermassen 
geschrieben: $\forall x:P(x)$
\vspace{5mm}

Das Symbol $\forall$ wird auch Allquantor oder Generalisator oder 
Generalisator-Operator genannt.
\vspace{5mm}

\subsubsection{Existenzquantor}

Wenn es mindestens ein $x$ gibt, also: «es gibt (mind.) ein $x$, so 
dass $P(x)$ gilt», dann schreibt man: $\exists x:P(x)$
\vspace{5mm}

Das Symbol $\exists$ wird Existenzquantor oder Partikularisator genannt.
\vspace{5mm}

\subsubsection{Eindeutigkeitsquantor}

Wenn es jedoch genau nur ein Objekt gibt, dann: «es gibt genau ein $x$, 
so dass $P(x)$ gilt». Dies wird wie folgt dargestellt: $\exists! x:P(x)$.
\vspace{5mm}

Das Symbol $\exists!$ wird Eindeutigkeitsquantor oder Einzigkeitsquantor 
genannt.
\vspace{5mm}

\textbf{Ausserdem:} Ein Quantor bindet stärker als alle anderen logischen 
Operatoren!!!

\subsection{Rechenregeln}

\begin{align}
\fbox{\begin{minipage}{11.8cm}
    \textbf{Beispiel für die Verneinung}:
    $\neg \forall xPx\leftrightarrow \exists x\neg Px$: 
    Die Aussage: Alle Wände sind weiss läst sich auch als "Nicht alle Wände 
    sind weiss", und als "Es gibt mindestens eine Wand, die nicht weiss ist".
\end{minipage}}
\end{align}

\def\arraystretch{1.5}
\begin{table}[ht]
\begin{tabular}[t]{lll}
    \hline
    Name & Regel\\
    \hline
    Verneinungssätze & $\neg(\forall x : A(x)) \iff \exists x : \neg A(x)$\\
    & $\neg(\forall x : \neg A(x)) \iff \exists x : A(x)$\\
    & $\neg(\exists x : A(x)) \iff \forall x : \neg A(x)$\\
    & $\neg(\exists x : \neg A(x)) \iff \forall x : A(x)$\\
    Vertauschbarkeitssätze & $\forall x, y : A(x, y) \iff \forall y, x : A(x, y)$\\
    & $\exists x, y : A(x, y) \iff \exists y, x : A(x, y)$\\
    & $\exists x \forall y : A(x, y) \color{red}\Rightarrow \color{black}\forall y \exists x : A(x, y)$\\
    & $\forall x : A(x) \color{red}\Rightarrow \color{black}\exists x : A(x)$\\
    & $\exists! x : A(x) \color{red}\Rightarrow \color{black}\exists x : A(x)$\\
    \hline
\end{tabular}
\end{table}

\begin{align}
\fbox{\begin{minipage}{11.8cm}
    \textbf{Beispiel Negation}:\\
    Es gilt nicht, dass es eine Farbe $z$ gibt, so dass alle Menschen $x$ 
    ein Auto $y$ besitzen mit der Farbe $z$.\\
    \\Formel: $\neg(\exists z \forall x \exists y : A(x, y, z))$
    \\3. Verneinungssatz: $\iff \forall z: \neg(\forall x \exists y : A(x, y, z))$
    \\1. Verneinungssatz: $\iff \forall z \exists x : \neg(\exists y: A(x, y, z))$
    \\3. Verneinungssatz: $\iff \forall z \exists x \forall y : \neg A(x, y, z)$
    \\
    \\"Für jede Farbe $z$ gilt, dass es eine Person $x$ gibt, so dass alle ihre Autos nicht Farbe $z$ haben."
\end{minipage}}
\end{align}
